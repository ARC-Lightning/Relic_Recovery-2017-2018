\documentclass[letterpaper]{article}
\usepackage{hyperref}
% \usepackage{algorithm}
% \usepackage{algpseudocode}

\begin{document}
\title{The Lightning Codebase}
\author{FTC Team 4410 (Lightning)}
\date{November 2017}
\maketitle

\begin{abstract}
The Lightning Codebase for 2017-2018 FTC Relic Recovery contains many unique features that are new to the team and uphold rigorous principles. With Kotlin, the development and maintainability of this codebase is vastly improved compared to Java. By splitting code into coherent modules, separation of responsibility and abstraction are achieved while enhancing readability. With a vector-based implementation of Mecanum wheel control, autonomous navigation is simplified. Inspired by the Markov Decision Process (MDP), decisions during autonomous are designed to be made automatically by the robot, according to a few given parameters. Most of all, this codebase strives to be maintainable and extensible for future competitions by enforcing documentation.
\end{abstract}
\section{Style and Language}

\subsection{Kotlin}

\subsubsection{Definition}
Kotlin is a programming language that runs on the Java Virtual Machine (JVM) platform. It has been supported by Google as an official language for Android since May 2017. It offers cleaner and more concise syntax, support for multiple programming paradigms \textit{(such as functional)}, and is fully interoperable with Java. It is open source under the Apache 2.0 license. Google's introduction to Kotlin is located at: \texttt{https://developer.android.com/kotlin/index.html}

\subsubsection{Comparison with Java}
% Discuss: lambdas, null safety!, string templates, singletons, data classes, operator overloading, collection mutability (and mutability in general)
\begin{description}

\item[Lambdas]
Whereas Java offers Runnables as an equivalent to a lambda with no parameters or return values\footnote{We ignored Java 8+ because FTC's official Robot Controller repository, at the time of writing, still uses Java 7.}, Kotlin offers full lambda support with parameters and implicit return values. This allows us to implement callback mappings in a readable, writable, and understandable way. For example, the \texttt{GamepadListener} class contains a \texttt{Map} from the \texttt{typealias} \texttt{GamepadProperty} to the \texttt{typealias} \texttt{GamepadRule}, both of which represent lambdas with a given set of input types. Lambdas are also extensively used in functional iterator transformations such as \texttt{map}, \texttt{filter}, and \texttt{reduce}. Kotlin's support for Lambdas and these methods makes implementations of these functions possible without third-party libraries, as opposed using Java 7.

\item[Null Safety]
Types in Kotlin are non-nullable by default, whereas any non-primitive type in Java can be \texttt{null}. Although this does not bring exciting new features to our code, it encourages us to handle null cases when the type is declared to be nullable, preventing \texttt{NullPointerException}s.

\item[Data Classes]
Data classes in Kotlin were designed to address use cases where the only responsibility of a class is to hold data. It implicitly provides functions such as \texttt{equals()} and \texttt{toString()} that are specialized for data. It eliminates the getter/setter pattern in Java, which produces plenty of boilerplate code. Data classes are used for the \texttt{Position} data structure in ACS, which includes a two-dimensional coordinate and an orientation in radians. Before the conversion to Kotlin, getters and setters were implemented for each field, using up to 50 lines\footnote{Including documentation but without the header (JavaDoc before the class declaration)}. The converted code only needed 1 concise line.

\end{description}

There are many other delightful features in Kotlin that boost developer productivity and satisfaction compared to Java. Common idioms in Kotlin are demonstrated at: \texttt{https://kotlinlang.org/docs/reference/idioms.html}

\subsubsection{Low Entry Point}

Kotlin is developed by JetBrains. The platform on which Android Studio is based on, IntelliJ, is also developed by JetBrains. JetBrains is known for superb language support in IDEs, and Kotlin is no exception. IntelliJ's Kotlin plugin is capable of converting any Java file directly to Kotlin just by copy-and-paste, so developers on the team may choose to write in either language. Although consistency in language is encouraged, Kotlin's seamless support for Java inter-op allows our developers to also keep their code in Java---no practical disadvantage results from the mixing of both languages.

\subsection{Style Guide}

To ensure consistency in readability and formatting, our codebase is fully compliant with \href{https://android.github.io/kotlin-guides/style.html}{Google's Android Kotlin Style Guide}. Strict adherence to a style guide provides us with consistent code style that does not obstruct productivity and improves readability.

\subsubsection{Interfaces}

The codebase is split into well-defined modules that correspond to the individual subsystems present in the robot. For example, The \texttt{IGlyphManipulator} interface allows other parts of the code to access the output devices that control the glyph mechanism; the \texttt{IDrivetrain} interface provides vector-based abstraction of movement and rotation using the drivetrain. For each module there is an interface file (prefixed with \texttt{I}) and an arbitrary amount of classes that implement that interface. This separation of protocol and implementation was intended for independency of specific implementations, enabling it to be swapped for another without much disturbance to other parts of the codebase. With this capability, seamless integration of code across authors can be achieved.

\subsubsection{Responsibilities}

Each module in the codebase has specific responsibilities, such as operating the jewel knocker or moving the robot. This form of organization facilitates the location of bugs in the code and aligns with our principles, which are discussed below.

\subsection{Principles}

It is our belief that compliance with a set of well-regarded programming principles can drastically improve code maintainability and quality in general. It is anticipated that this codebase might be inherited by our team in upcoming years, and so the intelligibility and reusability of the code is crucial to future success. Below is an enumeration of the principles with which this codebase attempts to comply.
\begin{description}

\item[Minimizing Coupling]
Coupling can be roughly defined as the amount of interconnection between modules. Minimizing coupling increases the level of independence of each module and limits the sharing of data between them, making data and logic more self-sufficient. By doing so, modules can be maintained individually; logic can be modified without disrupting other modules unless changes are made to the interface. Furthermore, reducing coupling improves readability by minimizing the amount of inter-modular references, thereby making code more focused and straightforward.

\item[D.R.Y.]
\textit{D.R.Y.} stands for "Don't Repeat Yourself." Its adherence encourages the developer to reduce redundancy throughout the system through techniques such as creating utility functions (and thereby reducing average complexity per function), creating universal parameters to store configurable data, and eliminating extraneous or redundant computations. This technique provides considerable benefits to the maintainer, the reader, and the writer. Parameters that require fine-tuning only need to be altered in one location in the code; readers and writers can better understand the purpose and the implementation of each function, increasing abstraction and expanding the capacity of logical complexity.

\item[Simplicity]
There is no practical reason to artificially complicate code. Not only is this act counterproductive, it impedes future understanding and expansion. The quality of this project should be inversely proportional to the total amount of lines, given that the desired behaviors are fulfilled. Functions with high levels of abstraction should be optimized to follow natural language patterns such as prepositional phrases and adverbs. Ideal examples of this rule are demonstrated below.
\begin{center}
	\texttt{ToggleInput.from(gamepad1.x).controls(glypher.flywheel)}
	\texttt{sage.nextDecision(prefer = customSequence).perform()}
	\texttt{mecanum.drive((3, 3), Options.QUICKLY).then \{\dots\}}
\end{center}

\end{description}

\section{Technical Description}

\subsection{Drivetrain}
With Mecanum wheels, the robot is able to move in at least 8 directions from a given point without turning. With careful adjustments to individual motor powers, the robot can move in any possible direction expressible in two-dimensional Cartesian coordinates; however, the accuracy of this method is uncertain, and thus there is an option to disable this feature for autonomous navigation. The drivetrain subsystem should be optimized and designed for their operation. It should be designed to work with both TeleOp and Autonomous logic as effortlessly as possible; it should also provide plenty of flexibility for testing and experimentation.

\subsubsection{Use of Vectors}
This interface presents movements as vectors on a 2D plane, in which the positive $y$ direction represents forward and the positive $x$ direction represents right. Note that the plane is \textbf{not} projected relative to the playing field, but rather to the robot, i.e. the direction that the robot is facing is the positive $y$ direction. A collection of examples is presented below.
\begin{center}
	$(1, 1) \rightarrow$ forwards for 1 inch, right for 1 inch (diagonal)
	
	$(5, -2) \rightarrow$ forwards for 5 inches, left for 2 inches
	
	$(-5.5, 0) \rightarrow$ backwards for 5.5 inches
\end{center}

\subsubsection{Internal Logic}
Mecanum wheels are placed in four relative positions to form a rectangle. A rectangle has two diagonals. In Mecanum movements, motors at the end of each of the diagonals have the same power value. Following the D.R.Y. principle, only two scalar values are necessary as output, and therefore the function described here takes two inputs---the $x$ and $y$ components of the vector---and produces two outputs---the power value for each diagonal. The following enumeration describes the steps necessary to calculate the desired power value for each motor from any given vector:
\begin{description}
	\item[Rotation]
	We have observed a relationship between the components of vectors and their corresponding power values. After rotating the given vector $45^\circ$ clockwise, the two components correspond to the desired power of each diagonal pair of motors. Consider the following examples:
\begin{center}
	$(0, 1) \rightarrow (\frac{\sqrt{2}}{2}, \frac{\sqrt{2}}{2})$: To move in the direction $(0, 1)$ \textit{(i.e. forward)}, Both pairs of motors should have power values that are positive and equal.
	
	$(-1, -1) \rightarrow (-\sqrt{2},0)$: To move in the direction $(-1, -1)$ \textit{(i.e. left and backward)}, one pair should run in reverse and the other pair should be stationary.
\end{center}
	\item[Scaling of Power (Autonomous)]
	In autonomous, there is no guarantee that the input vector's components would be in range of $-1 \leq 0 \leq 1$, which applies to motor power, so the components are proportionally scaled to fit it. The algorithm used by this process is as follows:
	\begin{center}
		$f((x, y)) = (\frac{x}{\max(x, y)}, \frac{y}{\max(x, y)})$
	\end{center}
	This method of scaling is crucial to the robot's ability to move in more than 8 directions.
	
	\item[Assignment]
	According to the current configurations of the mecanum wheels, the $x$ component is assigned to the front left and rear right diagonal pair, and the $y$ component is assigned to the front right and rear left diagonal pair.
	
	\item[Setting Target Positions (Autonomous)]
	The components of the rotated but unscaled vector is converted from inches to motor ticks, which are then assigned to the motors with the mapping described above.
\end{description}

\subsubsection{Interface Methods}

\paragraph{Movement}
To support vector-based operations for autonomous, the drivetrain interface accepts any arbitrary vector from the origin as a parameter. It then applies desired movements to drivetrain motors to accomplish this task without turning the robot. An optional power multiplier parameter is provided for customization of speed. The movement methods designed for TeleOp are optimized for being controlled in a synchronous loop. They never block the main thread and override any existing power applied to the motors.

\paragraph{Turning}
Turning methods rotate the robot in place. Methods designed for autonomous receive the desired angle of rotation in radians of range $-2\pi < n < 2\pi$. Methods designed for TeleOp receive either a boolean denoting direction of rotation or a number denoting the desired turning speed.

% TODO actuation

\subsubsection{Driver Assistance}
In situations such as lining up with the cryptobox, drivetrain controls can be twitchy and unintuitive. The support for precise drivetrain movement has been deemed advantageous by our drivers, and thus they have been implemented as a mode. When this mode is engaged, all motor power outputs are reduced by a multiplier, thus scaling down the robot's overall velocity.

\subsection{Autonomous}
Lightning's autonomous routine revolves around the concept of Tasks, of which examples include reading the VuMark, knocking the correct jewel, and placing the preloaded glyph into the intended cryptobox. The \texttt{AutoNav} module is responsible for autonomous navigation, while the \texttt{DecisionMaker} module decides the order in which the tasks are performed.

\subsubsection{Decision Making}
Task descriptors include a priority factor and a reliability factor for the task described, both of which have a range of $0 \leq n \leq 1$. The priority factor corresponds to the amount of points gained from performing the task (i.e. the immediate reward in M.D.P. terms), and the reliability factor corresponds to the success rate of the task. An \texttt{ElapsedTime} timer is integrated into the \texttt{DecisionMaker}, which tracks how much time has passed since the autonomous period had started. A discount factor, which represents the rate at which delayed rewards decay, is dynamically calculated to be inversely proportional to the timer value's difference from 30 seconds; this means that the robot's preference of tasks with greater reward will increase as the autonomous period progresses. The algorithm implements the Markov Decision Process using these parameters.

In addition to automated decision-making, the code also supports a predefined sequence, declared in the \texttt{LinearOpMode} file.

\subsubsection{Navigation}
The overall demand for navigation in this year's autonomous tasks is low, so its implementation is relatively simple. Two adjustable vector parameters represent the location of the jewels and cryptobox columns relative to the starting point, and a column width parameter defines the distance between the loading positions for adjacent cryptobox columns. \texttt{AutoNav} provides functions that move the robot between positions defined by the vectors and the starting point. When the Autonomous period begins, the main routine requests an unfinished task from \texttt{DecisionMaker} to perform, during which the task routines use these \texttt{AutoNav} functions to navigate as necessary.

\subsection{Configuration}
To reupload the entire Robot Controller app for every parameter tweak has been terribly inefficient in our experience. Each upload takes roughly 1 minute to take place, and this delay greatly reduces the momentum of agile development. To respond to this problem, we have migrated our configurable parameters (discussed above in \textit{D.R.Y.}) to Java Property files in the phone's local storage. Upon the start of any OpMode, the parameters are reloaded from storage. Since uploading configuration files is practically instant, our momentum in rapid development is restored to its full potential. Depending on demand, we may continue to implement scripting for greater efficiency.

\subsubsection{Object-Oriented Model}
Each module implementation that uses configuration files has a nested class where all parameters are declared as class fields, which are initialized to the values read from storage through module \texttt{ConfigFile}. This module is built with abilities to cast the raw string value to a variety of common data types, including \texttt{double}, \texttt{int}, and \texttt{boolean}.

%\subsection{I/O Subsystems}
%Each physical subsystem on the robot corresponds to a dedicated module in the codebase, which contains references to I/O devices (\texttt{DcMotor}, \texttt{Servo}, etc.), initialization procedures, configurable parameters, and public functions that provide abstraction in manipulating the subsystem's devices. These functions enhance readability in other modules. These I/O subsystems assume full responsibility of the operation of their declared hardware devices, which eases maintenance. The following subsections describe each I/O subsystem in detail.

%\subsubsection{Glyph manipulator}
%This subsystem is responsible for the following hardware devices:
%\begin{description}
%	\item[The glyph-obtaining flywheels]
%	\item[Rods that put the glyph in place]
%	\item[Glyph propellers that allow the middle column to be finished]
%\end{description}

%\subsubsection{Jewel knocker}

% tests, 
\end{document}